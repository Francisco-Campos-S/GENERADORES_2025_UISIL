\begin{center}
    {\LARGE \textbf{II Examen de Matemáticas Financiera}}\\[1em]
    {\large \textbf{Estudiante:} {{ estudiante }}}
\end{center}

\vspace{1cm}

\textbf{Profesor: Francisco Campos Sandi}

\vspace{0.5cm}

\textbf{Instrucciones:} Resuelva cuidadosamente los siguientes problemas, mostrando todos los pasos. Entrega máxima sin excepciones. Fecha límite de entrega: \textbf{Miércoles 23 de Julio}.

\vspace{1cm}

\begin{enumerate}[leftmargin=*, label=\textbf{\arabic*.}]
  \item (10 puntos) La señora \textbf{ {{ nombre_persona1 }} } compró equipo por un valor de \textbf{\$\num{ {{ e1_monto }} }}. Acordó pagar en tres cuotas iguales a 4, 7 y 10 meses. Si la tasa de interés es del \textbf{ {{ e1_tasa }} }\% mensual, ¿cuánto debe pagar en cada cuota?

  \vspace{0.5cm}

  \item (10 puntos) La compañía \textbf{ {{ nombre_empresa1 }} } adquirió mercancía por \textbf{\$\num{ {{ e2_monto }} }} y planea pagar en tres pagos iguales a los 2, 5 y 12 meses. Si la tasa de interés es del \textbf{ {{ e2_tasa }} }\% mensual, ¿cuánto debe pagar en cada cuota?

  \vspace{0.5cm}

  \item (10 puntos) Un cliente firma un pagaré por \textbf{\$\num{ {{ e3_deuda1 }} }} a 5 meses, con una tasa anual del \textbf{ {{ e3_tasa }} }\%. Dos meses después contrae otra deuda por \textbf{\$\num{ {{ e3_deuda2 }} }} a 4 meses. Luego realiza un abono de \textbf{\$\num{ {{ e3_abono }} }} y acuerda pagar el saldo final 7 meses después del abono. ¿Cuál es el monto final a pagar?

  \vspace{0.5cm}

  \item (10 puntos) Se realiza un depósito de \textbf{₡\num{ {{ e4_deposito }} }} en una cuenta que ofrece interés compuesto anual del \textbf{ {{ e4_tasa }} }\%, capitalizable trimestralmente por \textbf{ {{ nombre_empresa2 }} }. ¿Cuál es el interés generado en un año?

  \vspace{0.5cm}

  \item (10 puntos) \textbf{ {{ nombre_persona2 }} } invierte \textbf{\$\num{ {{ e5_capital }} }} al \textbf{ {{ e5_tasa }} }\% anual durante 5 años. ¿Cuál será el interés total ganado? Considere interés compuesto.

  \vspace{0.5cm}

  \item (10 puntos) Se depositan \textbf{\$\num{ {{ e6_capital }} }} en una cuenta con una tasa de interés del \textbf{ {{ e6_tasa }} }\% mensual, capitalizable cada mes por \textbf{ {{ nombre_empresa3 }} }. ¿Cuál será el saldo acumulado después de 24 meses?
\end{enumerate}



