
\documentclass[10pt]{article}
\usepackage[T1]{fontenc}
\usepackage[utf8]{inputenc}
\usepackage[spanish]{babel}
\usepackage{lmodern}
\usepackage{amsmath, amssymb}
\usepackage{graphicx}
\usepackage{fancyhdr}
\usepackage{enumitem}
\usepackage{textcomp}
\pagestyle{fancy}
\fancyhf{}
\lhead{UISIL \\ ISB-24 INGENIERÍA EN SISTEMAS}
\rhead{\today}
\renewcommand{\headrulewidth}{1pt}
\renewcommand{\footrulewidth}{0.5pt}
\lfoot{\textbf{Elaborado por: Francisco Campos S.}}
\cfoot{\thepage}
\setlength{\headheight}{22.5pt}
\begin{document}
\begin{center}
    {\LARGE \textbf{II Examen de Matemáticas Financiera}}\\[1em]
    {\large \textbf{Estudiante:} ARIAS MARIN KENDY DAYANA}
\end{center}

\vspace{1cm}

\textbf{Profesor: Francisco Campos Sandi}

\vspace{0.5cm}

\textbf{Instrucciones:} Resuelva cuidadosamente los siguientes problemas, mostrando todos los pasos. Entrega máxima sin excepciones. Fecha límite de entrega: \textbf{Miércoles 23 de Julio}.

\vspace{1cm}

\begin{enumerate}[leftmargin=*, label=\textbf{\arabic*.}]
  \item (10 puntos) La señora \textbf{ Ana Vargas } compró equipo por un valor de \textbf{\$\num{ 23,575.00 }}. Acordó pagar en tres cuotas iguales a 4, 7 y 10 meses. Si la tasa de interés es del \textbf{ 3.1 }\% mensual, ¿cuánto debe pagar en cada cuota?

  \vspace{0.5cm}

  \item (10 puntos) La compañía \textbf{ Comercial ABC } adquirió mercancía por \textbf{\$\num{ 16,489.00 }} y planea pagar en tres pagos iguales a los 2, 5 y 12 meses. Si la tasa de interés es del \textbf{ 3.6 }\% mensual, ¿cuánto debe pagar en cada cuota?

  \vspace{0.5cm}

  \item (10 puntos) Un cliente firma un pagaré por \textbf{\$\num{ 7,964.00 }} a 5 meses, con una tasa anual del \textbf{ 34.0 }\%. Dos meses después contrae otra deuda por \textbf{\$\num{ 9,623.00 }} a 4 meses. Luego realiza un abono de \textbf{\$\num{ 4,855.00 }} y acuerda pagar el saldo final 7 meses después del abono. ¿Cuál es el monto final a pagar?

  \vspace{0.5cm}

  \item (10 puntos) Se realiza un depósito de \textbf{₡\num{ 155.084.00 }} en una cuenta que ofrece interés compuesto anual del \textbf{ 15.0 }\%, capitalizable trimestralmente por \textbf{ Comercial ABC }. ¿Cuál es el interés generado en un año?

  \vspace{0.5cm}

  \item (10 puntos) \textbf{ Carmen Rojas } invierte \textbf{\$\num{ 34,331.00 }} al \textbf{ 4.3 }\% anual durante 5 años. ¿Cuál será el interés total ganado? Considere interés compuesto.

  \vspace{0.5cm}

  \item (10 puntos) Se depositan \textbf{\$\num{ 61,429.00 }} en una cuenta con una tasa de interés del \textbf{ 2.4 }\% mensual, capitalizable cada mes por \textbf{ Tecnologías Avanzadas }. ¿Cuál será el saldo acumulado después de 24 meses?
\end{enumerate}



\newpage
\begin{center}
    {\LARGE \textbf{II Examen de Matemáticas Financiera}}\\[1em]
    {\large \textbf{Estudiante:} ARIAS MARIN KENNETH JESUS}
\end{center}

\vspace{1cm}

\textbf{Profesor: Francisco Campos Sandi}

\vspace{0.5cm}

\textbf{Instrucciones:} Resuelva cuidadosamente los siguientes problemas, mostrando todos los pasos. Entrega máxima sin excepciones. Fecha límite de entrega: \textbf{Miércoles 23 de Julio}.

\vspace{1cm}

\begin{enumerate}[leftmargin=*, label=\textbf{\arabic*.}]
  \item (10 puntos) La señora \textbf{ Eduardo Morales } compró equipo por un valor de \textbf{\$\num{ 22,799.00 }}. Acordó pagar en tres cuotas iguales a 4, 7 y 10 meses. Si la tasa de interés es del \textbf{ 2.8 }\% mensual, ¿cuánto debe pagar en cada cuota?

  \vspace{0.5cm}

  \item (10 puntos) La compañía \textbf{ Distribuidora Omega } adquirió mercancía por \textbf{\$\num{ 14,161.00 }} y planea pagar en tres pagos iguales a los 2, 5 y 12 meses. Si la tasa de interés es del \textbf{ 4.0 }\% mensual, ¿cuánto debe pagar en cada cuota?

  \vspace{0.5cm}

  \item (10 puntos) Un cliente firma un pagaré por \textbf{\$\num{ 7,947.00 }} a 5 meses, con una tasa anual del \textbf{ 29.0 }\%. Dos meses después contrae otra deuda por \textbf{\$\num{ 9,973.00 }} a 4 meses. Luego realiza un abono de \textbf{\$\num{ 3,535.00 }} y acuerda pagar el saldo final 7 meses después del abono. ¿Cuál es el monto final a pagar?

  \vspace{0.5cm}

  \item (10 puntos) Se realiza un depósito de \textbf{₡\num{ 144.174.00 }} en una cuenta que ofrece interés compuesto anual del \textbf{ 14.8 }\%, capitalizable trimestralmente por \textbf{ Distribuidora Omega }. ¿Cuál es el interés generado en un año?

  \vspace{0.5cm}

  \item (10 puntos) \textbf{ Jorge Sánchez } invierte \textbf{\$\num{ 31,462.00 }} al \textbf{ 4.6 }\% anual durante 5 años. ¿Cuál será el interés total ganado? Considere interés compuesto.

  \vspace{0.5cm}

  \item (10 puntos) Se depositan \textbf{\$\num{ 60,824.00 }} en una cuenta con una tasa de interés del \textbf{ 2.5 }\% mensual, capitalizable cada mes por \textbf{ Distribuidora Omega }. ¿Cuál será el saldo acumulado después de 24 meses?
\end{enumerate}



\newpage
\begin{center}
    {\LARGE \textbf{II Examen de Matemáticas Financiera}}\\[1em]
    {\large \textbf{Estudiante:} FERNANDEZ FLORES YOSEF SAID}
\end{center}

\vspace{1cm}

\textbf{Profesor: Francisco Campos Sandi}

\vspace{0.5cm}

\textbf{Instrucciones:} Resuelva cuidadosamente los siguientes problemas, mostrando todos los pasos. Entrega máxima sin excepciones. Fecha límite de entrega: \textbf{Miércoles 23 de Julio}.

\vspace{1cm}

\begin{enumerate}[leftmargin=*, label=\textbf{\arabic*.}]
  \item (10 puntos) La señora \textbf{ Eduardo Morales } compró equipo por un valor de \textbf{\$\num{ 24,446.00 }}. Acordó pagar en tres cuotas iguales a 4, 7 y 10 meses. Si la tasa de interés es del \textbf{ 3.1 }\% mensual, ¿cuánto debe pagar en cada cuota?

  \vspace{0.5cm}

  \item (10 puntos) La compañía \textbf{ Distribuidora Omega } adquirió mercancía por \textbf{\$\num{ 15,045.00 }} y planea pagar en tres pagos iguales a los 2, 5 y 12 meses. Si la tasa de interés es del \textbf{ 3.9 }\% mensual, ¿cuánto debe pagar en cada cuota?

  \vspace{0.5cm}

  \item (10 puntos) Un cliente firma un pagaré por \textbf{\$\num{ 7,470.00 }} a 5 meses, con una tasa anual del \textbf{ 27.2 }\%. Dos meses después contrae otra deuda por \textbf{\$\num{ 9,214.00 }} a 4 meses. Luego realiza un abono de \textbf{\$\num{ 3,377.00 }} y acuerda pagar el saldo final 7 meses después del abono. ¿Cuál es el monto final a pagar?

  \vspace{0.5cm}

  \item (10 puntos) Se realiza un depósito de \textbf{₡\num{ 147.998.00 }} en una cuenta que ofrece interés compuesto anual del \textbf{ 16.1 }\%, capitalizable trimestralmente por \textbf{ Servicios Globales }. ¿Cuál es el interés generado en un año?

  \vspace{0.5cm}

  \item (10 puntos) \textbf{ María Pérez } invierte \textbf{\$\num{ 34,610.00 }} al \textbf{ 4.3 }\% anual durante 5 años. ¿Cuál será el interés total ganado? Considere interés compuesto.

  \vspace{0.5cm}

  \item (10 puntos) Se depositan \textbf{\$\num{ 64,928.00 }} en una cuenta con una tasa de interés del \textbf{ 2.0 }\% mensual, capitalizable cada mes por \textbf{ Distribuidora Omega }. ¿Cuál será el saldo acumulado después de 24 meses?
\end{enumerate}



\newpage
\begin{center}
    {\LARGE \textbf{II Examen de Matemáticas Financiera}}\\[1em]
    {\large \textbf{Estudiante:} LARA HERNANDEZ DYLAN DANIEL}
\end{center}

\vspace{1cm}

\textbf{Profesor: Francisco Campos Sandi}

\vspace{0.5cm}

\textbf{Instrucciones:} Resuelva cuidadosamente los siguientes problemas, mostrando todos los pasos. Entrega máxima sin excepciones. Fecha límite de entrega: \textbf{Miércoles 23 de Julio}.

\vspace{1cm}

\begin{enumerate}[leftmargin=*, label=\textbf{\arabic*.}]
  \item (10 puntos) La señora \textbf{ Carmen Rojas } compró equipo por un valor de \textbf{\$\num{ 24,259.00 }}. Acordó pagar en tres cuotas iguales a 4, 7 y 10 meses. Si la tasa de interés es del \textbf{ 2.8 }\% mensual, ¿cuánto debe pagar en cada cuota?

  \vspace{0.5cm}

  \item (10 puntos) La compañía \textbf{ Importadora Central } adquirió mercancía por \textbf{\$\num{ 14,461.00 }} y planea pagar en tres pagos iguales a los 2, 5 y 12 meses. Si la tasa de interés es del \textbf{ 3.5 }\% mensual, ¿cuánto debe pagar en cada cuota?

  \vspace{0.5cm}

  \item (10 puntos) Un cliente firma un pagaré por \textbf{\$\num{ 6,893.00 }} a 5 meses, con una tasa anual del \textbf{ 31.8 }\%. Dos meses después contrae otra deuda por \textbf{\$\num{ 9,649.00 }} a 4 meses. Luego realiza un abono de \textbf{\$\num{ 4,865.00 }} y acuerda pagar el saldo final 7 meses después del abono. ¿Cuál es el monto final a pagar?

  \vspace{0.5cm}

  \item (10 puntos) Se realiza un depósito de \textbf{₡\num{ 159.592.00 }} en una cuenta que ofrece interés compuesto anual del \textbf{ 15.0 }\%, capitalizable trimestralmente por \textbf{ Distribuidora Omega }. ¿Cuál es el interés generado en un año?

  \vspace{0.5cm}

  \item (10 puntos) \textbf{ Jorge Sánchez } invierte \textbf{\$\num{ 33,620.00 }} al \textbf{ 5.2 }\% anual durante 5 años. ¿Cuál será el interés total ganado? Considere interés compuesto.

  \vspace{0.5cm}

  \item (10 puntos) Se depositan \textbf{\$\num{ 69,861.00 }} en una cuenta con una tasa de interés del \textbf{ 2.3 }\% mensual, capitalizable cada mes por \textbf{ Distribuidora Omega }. ¿Cuál será el saldo acumulado después de 24 meses?
\end{enumerate}



\newpage
\begin{center}
    {\LARGE \textbf{II Examen de Matemáticas Financiera}}\\[1em]
    {\large \textbf{Estudiante:} RUIZ ARAYA CRISTOPHER ANDREY}
\end{center}

\vspace{1cm}

\textbf{Profesor: Francisco Campos Sandi}

\vspace{0.5cm}

\textbf{Instrucciones:} Resuelva cuidadosamente los siguientes problemas, mostrando todos los pasos. Entrega máxima sin excepciones. Fecha límite de entrega: \textbf{Miércoles 23 de Julio}.

\vspace{1cm}

\begin{enumerate}[leftmargin=*, label=\textbf{\arabic*.}]
  \item (10 puntos) La señora \textbf{ Eduardo Morales } compró equipo por un valor de \textbf{\$\num{ 23,593.00 }}. Acordó pagar en tres cuotas iguales a 4, 7 y 10 meses. Si la tasa de interés es del \textbf{ 2.7 }\% mensual, ¿cuánto debe pagar en cada cuota?

  \vspace{0.5cm}

  \item (10 puntos) La compañía \textbf{ Distribuidora Omega } adquirió mercancía por \textbf{\$\num{ 15,768.00 }} y planea pagar en tres pagos iguales a los 2, 5 y 12 meses. Si la tasa de interés es del \textbf{ 3.5 }\% mensual, ¿cuánto debe pagar en cada cuota?

  \vspace{0.5cm}

  \item (10 puntos) Un cliente firma un pagaré por \textbf{\$\num{ 6,305.00 }} a 5 meses, con una tasa anual del \textbf{ 31.7 }\%. Dos meses después contrae otra deuda por \textbf{\$\num{ 10,534.00 }} a 4 meses. Luego realiza un abono de \textbf{\$\num{ 3,601.00 }} y acuerda pagar el saldo final 7 meses después del abono. ¿Cuál es el monto final a pagar?

  \vspace{0.5cm}

  \item (10 puntos) Se realiza un depósito de \textbf{₡\num{ 159.615.00 }} en una cuenta que ofrece interés compuesto anual del \textbf{ 16.6 }\%, capitalizable trimestralmente por \textbf{ Tecnologías Avanzadas }. ¿Cuál es el interés generado en un año?

  \vspace{0.5cm}

  \item (10 puntos) \textbf{ Jorge Sánchez } invierte \textbf{\$\num{ 34,805.00 }} al \textbf{ 4.6 }\% anual durante 5 años. ¿Cuál será el interés total ganado? Considere interés compuesto.

  \vspace{0.5cm}

  \item (10 puntos) Se depositan \textbf{\$\num{ 60,849.00 }} en una cuenta con una tasa de interés del \textbf{ 2.2 }\% mensual, capitalizable cada mes por \textbf{ Tecnologías Avanzadas }. ¿Cuál será el saldo acumulado después de 24 meses?
\end{enumerate}



\newpage
\begin{center}
    {\LARGE \textbf{II Examen de Matemáticas Financiera}}\\[1em]
    {\large \textbf{Estudiante:} VARGAS GUTIERREZ JAIRO}
\end{center}

\vspace{1cm}

\textbf{Profesor: Francisco Campos Sandi}

\vspace{0.5cm}

\textbf{Instrucciones:} Resuelva cuidadosamente los siguientes problemas, mostrando todos los pasos. Entrega máxima sin excepciones. Fecha límite de entrega: \textbf{Miércoles 23 de Julio}.

\vspace{1cm}

\begin{enumerate}[leftmargin=*, label=\textbf{\arabic*.}]
  \item (10 puntos) La señora \textbf{ María Pérez } compró equipo por un valor de \textbf{\$\num{ 22,460.00 }}. Acordó pagar en tres cuotas iguales a 4, 7 y 10 meses. Si la tasa de interés es del \textbf{ 2.5 }\% mensual, ¿cuánto debe pagar en cada cuota?

  \vspace{0.5cm}

  \item (10 puntos) La compañía \textbf{ Importadora Central } adquirió mercancía por \textbf{\$\num{ 15,057.00 }} y planea pagar en tres pagos iguales a los 2, 5 y 12 meses. Si la tasa de interés es del \textbf{ 3.4 }\% mensual, ¿cuánto debe pagar en cada cuota?

  \vspace{0.5cm}

  \item (10 puntos) Un cliente firma un pagaré por \textbf{\$\num{ 6,125.00 }} a 5 meses, con una tasa anual del \textbf{ 34.8 }\%. Dos meses después contrae otra deuda por \textbf{\$\num{ 9,316.00 }} a 4 meses. Luego realiza un abono de \textbf{\$\num{ 3,338.00 }} y acuerda pagar el saldo final 7 meses después del abono. ¿Cuál es el monto final a pagar?

  \vspace{0.5cm}

  \item (10 puntos) Se realiza un depósito de \textbf{₡\num{ 151.673.00 }} en una cuenta que ofrece interés compuesto anual del \textbf{ 16.3 }\%, capitalizable trimestralmente por \textbf{ Servicios Globales }. ¿Cuál es el interés generado en un año?

  \vspace{0.5cm}

  \item (10 puntos) \textbf{ Carmen Rojas } invierte \textbf{\$\num{ 31,724.00 }} al \textbf{ 4.9 }\% anual durante 5 años. ¿Cuál será el interés total ganado? Considere interés compuesto.

  \vspace{0.5cm}

  \item (10 puntos) Se depositan \textbf{\$\num{ 66,354.00 }} en una cuenta con una tasa de interés del \textbf{ 2.1 }\% mensual, capitalizable cada mes por \textbf{ Servicios Globales }. ¿Cuál será el saldo acumulado después de 24 meses?
\end{enumerate}



\newpage
\begin{center}
    {\LARGE \textbf{II Examen de Matemáticas Financiera}}\\[1em]
    {\large \textbf{Estudiante:} VARGAS GUTIERREZ JOSUE ANDRES}
\end{center}

\vspace{1cm}

\textbf{Profesor: Francisco Campos Sandi}

\vspace{0.5cm}

\textbf{Instrucciones:} Resuelva cuidadosamente los siguientes problemas, mostrando todos los pasos. Entrega máxima sin excepciones. Fecha límite de entrega: \textbf{Miércoles 23 de Julio}.

\vspace{1cm}

\begin{enumerate}[leftmargin=*, label=\textbf{\arabic*.}]
  \item (10 puntos) La señora \textbf{ Carmen Rojas } compró equipo por un valor de \textbf{\$\num{ 24,913.00 }}. Acordó pagar en tres cuotas iguales a 4, 7 y 10 meses. Si la tasa de interés es del \textbf{ 3.2 }\% mensual, ¿cuánto debe pagar en cada cuota?

  \vspace{0.5cm}

  \item (10 puntos) La compañía \textbf{ Importadora Central } adquirió mercancía por \textbf{\$\num{ 15,363.00 }} y planea pagar en tres pagos iguales a los 2, 5 y 12 meses. Si la tasa de interés es del \textbf{ 3.5 }\% mensual, ¿cuánto debe pagar en cada cuota?

  \vspace{0.5cm}

  \item (10 puntos) Un cliente firma un pagaré por \textbf{\$\num{ 7,263.00 }} a 5 meses, con una tasa anual del \textbf{ 33.1 }\%. Dos meses después contrae otra deuda por \textbf{\$\num{ 9,561.00 }} a 4 meses. Luego realiza un abono de \textbf{\$\num{ 3,046.00 }} y acuerda pagar el saldo final 7 meses después del abono. ¿Cuál es el monto final a pagar?

  \vspace{0.5cm}

  \item (10 puntos) Se realiza un depósito de \textbf{₡\num{ 148.677.00 }} en una cuenta que ofrece interés compuesto anual del \textbf{ 17.7 }\%, capitalizable trimestralmente por \textbf{ Servicios Globales }. ¿Cuál es el interés generado en un año?

  \vspace{0.5cm}

  \item (10 puntos) \textbf{ María Pérez } invierte \textbf{\$\num{ 34,487.00 }} al \textbf{ 5.4 }\% anual durante 5 años. ¿Cuál será el interés total ganado? Considere interés compuesto.

  \vspace{0.5cm}

  \item (10 puntos) Se depositan \textbf{\$\num{ 64,643.00 }} en una cuenta con una tasa de interés del \textbf{ 2.0 }\% mensual, capitalizable cada mes por \textbf{ Importadora Central }. ¿Cuál será el saldo acumulado después de 24 meses?
\end{enumerate}



\newpage
\end{document}