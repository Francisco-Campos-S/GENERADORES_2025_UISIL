
\documentclass[10pt]{article}
\usepackage[T1]{fontenc}
\usepackage[utf8]{inputenc}
\usepackage[spanish]{babel}
\usepackage{lmodern}
\usepackage{amsmath, amssymb}
\usepackage{graphicx}
\usepackage{fancyhdr}
\usepackage{enumitem}
\usepackage{textcomp}
\pagestyle{fancy}
\fancyhf{}
\lhead{UISIL \\ ISB-24 INGENIERÍA EN SISTEMAS}
\rhead{\today}
\renewcommand{\headrulewidth}{1pt}
\renewcommand{\footrulewidth}{0.5pt}
\lfoot{\textbf{Elaborado por: Francisco Campos S.}}
\cfoot{\thepage}
\setlength{\headheight}{22.5pt}
\begin{document}
\begin{center}
    {\LARGE \textbf{II Examen de Matemáticas Financiera}}\\[1em]
    {\large \textbf{Estudiante:} ARIAS MARIN KENDY DAYANA}
\end{center}

\vspace{1cm}

\textbf{Profesor: Francisco Campos Sandi}

\vspace{0.5cm}

\textbf{Instrucciones:} Resuelva cuidadosamente los siguientes problemas, mostrando todos los pasos. Entrega máxima sin excepciones. Fecha límite de entrega: \textbf{Miércoles 23 de Julio}.

\vspace{1cm}

\begin{enumerate}[leftmargin=*, label=\textbf{\arabic*.}]
  \item (10 puntos) La señora \textbf{ Ana Vargas } compró equipo por un valor de \textbf{\$\num{ 25,791.00 }}. Acordó pagar en tres cuotas iguales a 4, 7 y 10 meses. Si la tasa de interés es del \textbf{ 3.2 }\% mensual, ¿cuánto debe pagar en cada cuota?

  \vspace{0.5cm}

  \item (10 puntos) La compañía \textbf{ Servicios Globales } adquirió mercancía por \textbf{\$\num{ 15,967.00 }} y planea pagar en tres pagos iguales a los 2, 5 y 12 meses. Si la tasa de interés es del \textbf{ 3.3 }\% mensual, ¿cuánto debe pagar en cada cuota?

  \vspace{0.5cm}

  \item (10 puntos) Un cliente firma un pagaré por \textbf{\$\num{ 6,670.00 }} a 5 meses, con una tasa anual del \textbf{ 34.5 }\%. Dos meses después contrae otra deuda por \textbf{\$\num{ 10,296.00 }} a 4 meses. Luego realiza un abono de \textbf{\$\num{ 3,072.00 }} y acuerda pagar el saldo final 7 meses después del abono. ¿Cuál es el monto final a pagar?

  \vspace{0.5cm}

  \item (10 puntos) Se realiza un depósito de \textbf{₡\num{ 158.103.00 }} en una cuenta que ofrece interés compuesto anual del \textbf{ 16.9 }\%, capitalizable trimestralmente por \textbf{ Servicios Globales }. ¿Cuál es el interés generado en un año?

  \vspace{0.5cm}

  \item (10 puntos) \textbf{ Jorge Sánchez } invierte \textbf{\$\num{ 33,021.00 }} al \textbf{ 4.7 }\% anual durante 5 años. ¿Cuál será el interés total ganado? Considere interés compuesto.

  \vspace{0.5cm}

  \item (10 puntos) Se depositan \textbf{\$\num{ 60,534.00 }} en una cuenta con una tasa de interés del \textbf{ 2.0 }\% mensual, capitalizable cada mes por \textbf{ Tecnologías Avanzadas }. ¿Cuál será el saldo acumulado después de 24 meses?
\end{enumerate}



\newpage
\begin{center}
    {\LARGE \textbf{II Examen de Matemáticas Financiera}}\\[1em]
    {\large \textbf{Estudiante:} ARIAS MARIN KENNETH JESUS}
\end{center}

\vspace{1cm}

\textbf{Profesor: Francisco Campos Sandi}

\vspace{0.5cm}

\textbf{Instrucciones:} Resuelva cuidadosamente los siguientes problemas, mostrando todos los pasos. Entrega máxima sin excepciones. Fecha límite de entrega: \textbf{Miércoles 23 de Julio}.

\vspace{1cm}

\begin{enumerate}[leftmargin=*, label=\textbf{\arabic*.}]
  \item (10 puntos) La señora \textbf{ Eduardo Morales } compró equipo por un valor de \textbf{\$\num{ 24,247.00 }}. Acordó pagar en tres cuotas iguales a 4, 7 y 10 meses. Si la tasa de interés es del \textbf{ 2.5 }\% mensual, ¿cuánto debe pagar en cada cuota?

  \vspace{0.5cm}

  \item (10 puntos) La compañía \textbf{ Comercial ABC } adquirió mercancía por \textbf{\$\num{ 15,565.00 }} y planea pagar en tres pagos iguales a los 2, 5 y 12 meses. Si la tasa de interés es del \textbf{ 3.8 }\% mensual, ¿cuánto debe pagar en cada cuota?

  \vspace{0.5cm}

  \item (10 puntos) Un cliente firma un pagaré por \textbf{\$\num{ 7,861.00 }} a 5 meses, con una tasa anual del \textbf{ 25.1 }\%. Dos meses después contrae otra deuda por \textbf{\$\num{ 9,743.00 }} a 4 meses. Luego realiza un abono de \textbf{\$\num{ 4,199.00 }} y acuerda pagar el saldo final 7 meses después del abono. ¿Cuál es el monto final a pagar?

  \vspace{0.5cm}

  \item (10 puntos) Se realiza un depósito de \textbf{₡\num{ 142.052.00 }} en una cuenta que ofrece interés compuesto anual del \textbf{ 15.8 }\%, capitalizable trimestralmente por \textbf{ Importadora Central }. ¿Cuál es el interés generado en un año?

  \vspace{0.5cm}

  \item (10 puntos) \textbf{ María Pérez } invierte \textbf{\$\num{ 33,874.00 }} al \textbf{ 4.8 }\% anual durante 5 años. ¿Cuál será el interés total ganado? Considere interés compuesto.

  \vspace{0.5cm}

  \item (10 puntos) Se depositan \textbf{\$\num{ 69,010.00 }} en una cuenta con una tasa de interés del \textbf{ 2.4 }\% mensual, capitalizable cada mes por \textbf{ Tecnologías Avanzadas }. ¿Cuál será el saldo acumulado después de 24 meses?
\end{enumerate}



\newpage
\begin{center}
    {\LARGE \textbf{II Examen de Matemáticas Financiera}}\\[1em]
    {\large \textbf{Estudiante:} FERNANDEZ FLORES YOSEF SAID}
\end{center}

\vspace{1cm}

\textbf{Profesor: Francisco Campos Sandi}

\vspace{0.5cm}

\textbf{Instrucciones:} Resuelva cuidadosamente los siguientes problemas, mostrando todos los pasos. Entrega máxima sin excepciones. Fecha límite de entrega: \textbf{Miércoles 23 de Julio}.

\vspace{1cm}

\begin{enumerate}[leftmargin=*, label=\textbf{\arabic*.}]
  \item (10 puntos) La señora \textbf{ Eduardo Morales } compró equipo por un valor de \textbf{\$\num{ 24,067.00 }}. Acordó pagar en tres cuotas iguales a 4, 7 y 10 meses. Si la tasa de interés es del \textbf{ 2.9 }\% mensual, ¿cuánto debe pagar en cada cuota?

  \vspace{0.5cm}

  \item (10 puntos) La compañía \textbf{ Importadora Central } adquirió mercancía por \textbf{\$\num{ 14,335.00 }} y planea pagar en tres pagos iguales a los 2, 5 y 12 meses. Si la tasa de interés es del \textbf{ 3.8 }\% mensual, ¿cuánto debe pagar en cada cuota?

  \vspace{0.5cm}

  \item (10 puntos) Un cliente firma un pagaré por \textbf{\$\num{ 6,656.00 }} a 5 meses, con una tasa anual del \textbf{ 30.9 }\%. Dos meses después contrae otra deuda por \textbf{\$\num{ 9,603.00 }} a 4 meses. Luego realiza un abono de \textbf{\$\num{ 3,522.00 }} y acuerda pagar el saldo final 7 meses después del abono. ¿Cuál es el monto final a pagar?

  \vspace{0.5cm}

  \item (10 puntos) Se realiza un depósito de \textbf{₡\num{ 153.822.00 }} en una cuenta que ofrece interés compuesto anual del \textbf{ 15.7 }\%, capitalizable trimestralmente por \textbf{ Servicios Globales }. ¿Cuál es el interés generado en un año?

  \vspace{0.5cm}

  \item (10 puntos) \textbf{ Carmen Rojas } invierte \textbf{\$\num{ 31,655.00 }} al \textbf{ 4.3 }\% anual durante 5 años. ¿Cuál será el interés total ganado? Considere interés compuesto.

  \vspace{0.5cm}

  \item (10 puntos) Se depositan \textbf{\$\num{ 60,348.00 }} en una cuenta con una tasa de interés del \textbf{ 2.3 }\% mensual, capitalizable cada mes por \textbf{ Comercial ABC }. ¿Cuál será el saldo acumulado después de 24 meses?
\end{enumerate}



\newpage
\begin{center}
    {\LARGE \textbf{II Examen de Matemáticas Financiera}}\\[1em]
    {\large \textbf{Estudiante:} LARA HERNANDEZ DYLAN DANIEL}
\end{center}

\vspace{1cm}

\textbf{Profesor: Francisco Campos Sandi}

\vspace{0.5cm}

\textbf{Instrucciones:} Resuelva cuidadosamente los siguientes problemas, mostrando todos los pasos. Entrega máxima sin excepciones. Fecha límite de entrega: \textbf{Miércoles 23 de Julio}.

\vspace{1cm}

\begin{enumerate}[leftmargin=*, label=\textbf{\arabic*.}]
  \item (10 puntos) La señora \textbf{ Eduardo Morales } compró equipo por un valor de \textbf{\$\num{ 25,237.00 }}. Acordó pagar en tres cuotas iguales a 4, 7 y 10 meses. Si la tasa de interés es del \textbf{ 2.8 }\% mensual, ¿cuánto debe pagar en cada cuota?

  \vspace{0.5cm}

  \item (10 puntos) La compañía \textbf{ Importadora Central } adquirió mercancía por \textbf{\$\num{ 16,415.00 }} y planea pagar en tres pagos iguales a los 2, 5 y 12 meses. Si la tasa de interés es del \textbf{ 3.7 }\% mensual, ¿cuánto debe pagar en cada cuota?

  \vspace{0.5cm}

  \item (10 puntos) Un cliente firma un pagaré por \textbf{\$\num{ 7,819.00 }} a 5 meses, con una tasa anual del \textbf{ 32.1 }\%. Dos meses después contrae otra deuda por \textbf{\$\num{ 9,504.00 }} a 4 meses. Luego realiza un abono de \textbf{\$\num{ 4,613.00 }} y acuerda pagar el saldo final 7 meses después del abono. ¿Cuál es el monto final a pagar?

  \vspace{0.5cm}

  \item (10 puntos) Se realiza un depósito de \textbf{₡\num{ 151.120.00 }} en una cuenta que ofrece interés compuesto anual del \textbf{ 17.8 }\%, capitalizable trimestralmente por \textbf{ Comercial ABC }. ¿Cuál es el interés generado en un año?

  \vspace{0.5cm}

  \item (10 puntos) \textbf{ Jorge Sánchez } invierte \textbf{\$\num{ 32,526.00 }} al \textbf{ 4.4 }\% anual durante 5 años. ¿Cuál será el interés total ganado? Considere interés compuesto.

  \vspace{0.5cm}

  \item (10 puntos) Se depositan \textbf{\$\num{ 65,257.00 }} en una cuenta con una tasa de interés del \textbf{ 2.2 }\% mensual, capitalizable cada mes por \textbf{ Comercial ABC }. ¿Cuál será el saldo acumulado después de 24 meses?
\end{enumerate}



\newpage
\begin{center}
    {\LARGE \textbf{II Examen de Matemáticas Financiera}}\\[1em]
    {\large \textbf{Estudiante:} RUIZ ARAYA CRISTOPHER ANDREY}
\end{center}

\vspace{1cm}

\textbf{Profesor: Francisco Campos Sandi}

\vspace{0.5cm}

\textbf{Instrucciones:} Resuelva cuidadosamente los siguientes problemas, mostrando todos los pasos. Entrega máxima sin excepciones. Fecha límite de entrega: \textbf{Miércoles 23 de Julio}.

\vspace{1cm}

\begin{enumerate}[leftmargin=*, label=\textbf{\arabic*.}]
  \item (10 puntos) La señora \textbf{ Jorge Sánchez } compró equipo por un valor de \textbf{\$\num{ 25,128.00 }}. Acordó pagar en tres cuotas iguales a 4, 7 y 10 meses. Si la tasa de interés es del \textbf{ 3.5 }\% mensual, ¿cuánto debe pagar en cada cuota?

  \vspace{0.5cm}

  \item (10 puntos) La compañía \textbf{ Comercial ABC } adquirió mercancía por \textbf{\$\num{ 16,599.00 }} y planea pagar en tres pagos iguales a los 2, 5 y 12 meses. Si la tasa de interés es del \textbf{ 3.3 }\% mensual, ¿cuánto debe pagar en cada cuota?

  \vspace{0.5cm}

  \item (10 puntos) Un cliente firma un pagaré por \textbf{\$\num{ 7,535.00 }} a 5 meses, con una tasa anual del \textbf{ 32.1 }\%. Dos meses después contrae otra deuda por \textbf{\$\num{ 10,154.00 }} a 4 meses. Luego realiza un abono de \textbf{\$\num{ 3,360.00 }} y acuerda pagar el saldo final 7 meses después del abono. ¿Cuál es el monto final a pagar?

  \vspace{0.5cm}

  \item (10 puntos) Se realiza un depósito de \textbf{₡\num{ 147.138.00 }} en una cuenta que ofrece interés compuesto anual del \textbf{ 17.7 }\%, capitalizable trimestralmente por \textbf{ Importadora Central }. ¿Cuál es el interés generado en un año?

  \vspace{0.5cm}

  \item (10 puntos) \textbf{ Eduardo Morales } invierte \textbf{\$\num{ 31,295.00 }} al \textbf{ 5.4 }\% anual durante 5 años. ¿Cuál será el interés total ganado? Considere interés compuesto.

  \vspace{0.5cm}

  \item (10 puntos) Se depositan \textbf{\$\num{ 65,856.00 }} en una cuenta con una tasa de interés del \textbf{ 2.3 }\% mensual, capitalizable cada mes por \textbf{ Distribuidora Omega }. ¿Cuál será el saldo acumulado después de 24 meses?
\end{enumerate}



\newpage
\begin{center}
    {\LARGE \textbf{II Examen de Matemáticas Financiera}}\\[1em]
    {\large \textbf{Estudiante:} VARGAS GUTIERREZ JAIRO}
\end{center}

\vspace{1cm}

\textbf{Profesor: Francisco Campos Sandi}

\vspace{0.5cm}

\textbf{Instrucciones:} Resuelva cuidadosamente los siguientes problemas, mostrando todos los pasos. Entrega máxima sin excepciones. Fecha límite de entrega: \textbf{Miércoles 23 de Julio}.

\vspace{1cm}

\begin{enumerate}[leftmargin=*, label=\textbf{\arabic*.}]
  \item (10 puntos) La señora \textbf{ Ana Vargas } compró equipo por un valor de \textbf{\$\num{ 25,589.00 }}. Acordó pagar en tres cuotas iguales a 4, 7 y 10 meses. Si la tasa de interés es del \textbf{ 3.4 }\% mensual, ¿cuánto debe pagar en cada cuota?

  \vspace{0.5cm}

  \item (10 puntos) La compañía \textbf{ Comercial ABC } adquirió mercancía por \textbf{\$\num{ 14,879.00 }} y planea pagar en tres pagos iguales a los 2, 5 y 12 meses. Si la tasa de interés es del \textbf{ 3.8 }\% mensual, ¿cuánto debe pagar en cada cuota?

  \vspace{0.5cm}

  \item (10 puntos) Un cliente firma un pagaré por \textbf{\$\num{ 6,607.00 }} a 5 meses, con una tasa anual del \textbf{ 33.2 }\%. Dos meses después contrae otra deuda por \textbf{\$\num{ 10,825.00 }} a 4 meses. Luego realiza un abono de \textbf{\$\num{ 4,820.00 }} y acuerda pagar el saldo final 7 meses después del abono. ¿Cuál es el monto final a pagar?

  \vspace{0.5cm}

  \item (10 puntos) Se realiza un depósito de \textbf{₡\num{ 144.039.00 }} en una cuenta que ofrece interés compuesto anual del \textbf{ 16.7 }\%, capitalizable trimestralmente por \textbf{ Comercial ABC }. ¿Cuál es el interés generado en un año?

  \vspace{0.5cm}

  \item (10 puntos) \textbf{ Eduardo Morales } invierte \textbf{\$\num{ 33,622.00 }} al \textbf{ 5.1 }\% anual durante 5 años. ¿Cuál será el interés total ganado? Considere interés compuesto.

  \vspace{0.5cm}

  \item (10 puntos) Se depositan \textbf{\$\num{ 67,687.00 }} en una cuenta con una tasa de interés del \textbf{ 2.4 }\% mensual, capitalizable cada mes por \textbf{ Distribuidora Omega }. ¿Cuál será el saldo acumulado después de 24 meses?
\end{enumerate}



\newpage
\begin{center}
    {\LARGE \textbf{II Examen de Matemáticas Financiera}}\\[1em]
    {\large \textbf{Estudiante:} VARGAS GUTIERREZ JOSUE ANDRES}
\end{center}

\vspace{1cm}

\textbf{Profesor: Francisco Campos Sandi}

\vspace{0.5cm}

\textbf{Instrucciones:} Resuelva cuidadosamente los siguientes problemas, mostrando todos los pasos. Entrega máxima sin excepciones. Fecha límite de entrega: \textbf{Miércoles 23 de Julio}.

\vspace{1cm}

\begin{enumerate}[leftmargin=*, label=\textbf{\arabic*.}]
  \item (10 puntos) La señora \textbf{ Jorge Sánchez } compró equipo por un valor de \textbf{\$\num{ 24,157.00 }}. Acordó pagar en tres cuotas iguales a 4, 7 y 10 meses. Si la tasa de interés es del \textbf{ 3.3 }\% mensual, ¿cuánto debe pagar en cada cuota?

  \vspace{0.5cm}

  \item (10 puntos) La compañía \textbf{ Importadora Central } adquirió mercancía por \textbf{\$\num{ 14,095.00 }} y planea pagar en tres pagos iguales a los 2, 5 y 12 meses. Si la tasa de interés es del \textbf{ 3.8 }\% mensual, ¿cuánto debe pagar en cada cuota?

  \vspace{0.5cm}

  \item (10 puntos) Un cliente firma un pagaré por \textbf{\$\num{ 6,635.00 }} a 5 meses, con una tasa anual del \textbf{ 31.9 }\%. Dos meses después contrae otra deuda por \textbf{\$\num{ 10,657.00 }} a 4 meses. Luego realiza un abono de \textbf{\$\num{ 3,892.00 }} y acuerda pagar el saldo final 7 meses después del abono. ¿Cuál es el monto final a pagar?

  \vspace{0.5cm}

  \item (10 puntos) Se realiza un depósito de \textbf{₡\num{ 155.548.00 }} en una cuenta que ofrece interés compuesto anual del \textbf{ 14.4 }\%, capitalizable trimestralmente por \textbf{ Importadora Central }. ¿Cuál es el interés generado en un año?

  \vspace{0.5cm}

  \item (10 puntos) \textbf{ Ana Vargas } invierte \textbf{\$\num{ 33,700.00 }} al \textbf{ 4.4 }\% anual durante 5 años. ¿Cuál será el interés total ganado? Considere interés compuesto.

  \vspace{0.5cm}

  \item (10 puntos) Se depositan \textbf{\$\num{ 62,420.00 }} en una cuenta con una tasa de interés del \textbf{ 2.3 }\% mensual, capitalizable cada mes por \textbf{ Servicios Globales }. ¿Cuál será el saldo acumulado después de 24 meses?
\end{enumerate}



\newpage
\end{document}