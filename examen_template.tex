
\begin{center}
    {\LARGE \textbf{II Examen de Matemática Financiera}}\\[1em]
    {\large \textbf{Estudiante:} {{ estudiante }}}
\end{center}

\vspace{1cm}

\textbf{Profesor: Francisco Campos Sandi}

\textbf{Instrucciones:} Desarrolle con orden y claridad los siguientes ejercicios. Puede utilizar calculadora financiera. Fecha de entrega: \textbf{Miércoles 17 de Julio}.

\begin{enumerate}
  \item El señor Eduardo Morales realizó una compra de mercancía por \textbf{\$ {{ e1_monto }}}. Acordó hacer tres pagos iguales: en 3, 5 y 9 meses. Si la tasa de interés es del \textbf{{ e1_tasa }}\% mensual, ¿cuál será el valor de cada pago?

  \item La empresa Distribuidora Omega adquirió productos por \textbf{\$ {{ e2_monto }}} y planea pagarlos en tres cuotas iguales a 1, 6 y 11 meses. Si la tasa de interés es del \textbf{{ e2_tasa }}\% mensual, ¿cuánto deberá pagar en cada cuota?

  \item Una persona firma un pagaré por \textbf{\$ {{ e3_deuda1 }}} a pagar en 4 meses, con una tasa del \textbf{{ e3_tasa }}\% anual. A los 2 meses contrae otra deuda de \textbf{\$ {{ e3_deuda2 }}} a 3 meses. Luego, abona \textbf{\$ {{ e3_abono }}} y pacta el pago final a 6 meses del abono. ¿Cuánto debe pagar al final?

  \item Se depositan \textbf{₡{{ e4_deposito }}} en una cuenta con interés compuesto anual del \textbf{{ e4_tasa }}\%, convertible trimestralmente. ¿Cuál será el interés ganado al cabo de 1 año?

  \item Jorge invierte \textbf{\$ {{ e5_capital }}} al \textbf{{ e5_tasa }}\% anual durante 4 años. ¿Cuánto interés gana en ese periodo? Utilice interés compuesto.

  \item Se depositan \textbf{\$ {{ e6_capital }}} en una cuenta que paga \textbf{{ e6_tasa }}\% mensual, capitalizable mensualmente. ¿Cuál será el monto acumulado en 18 meses?
\end{enumerate}
