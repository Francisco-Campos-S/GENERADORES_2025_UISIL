
\documentclass[14pt]{article}
\usepackage[a4paper, left=2cm, right=2cm, top=2cm, bottom=2cm]{geometry}
\usepackage[utf8]{inputenc}
\usepackage[spanish]{babel}
\usepackage{amsmath, amssymb}
\usepackage{graphicx}
\usepackage{fancyhdr}
\pagestyle{fancy}
\fancyhf{}
\lhead{UISIL \\ ISB-24 INGENIERÍA EN SISTEMAS}
\rhead{\today}
\renewcommand{\headrulewidth}{1pt}
\renewcommand{\footrulewidth}{0.5pt}
\lfoot{\textbf{Elaborado por: Francisco Campos S.}}
\cfoot{\thepage}

\begin{document}

\begin{center}
    {\LARGE \textbf{II Examen de Matemática Financiera}}\\[1em]
    {\large \textbf{Estudiante:} ARIAS MARIN KENDY DAYANA}
\end{center}

\vspace{1cm}

\textbf{Profesor: Francisco Campos Sandi}

\textbf{Instrucciones:} Desarrolle con orden y claridad los siguientes ejercicios. Puede utilizar calculadora financiera. Fecha de entrega: \textbf{Miércoles 17 de Julio}.

\begin{enumerate}
  \item El señor Eduardo Morales realizó una compra de mercancía por \textbf{\$ 23,475.00}. Acordó hacer tres pagos iguales: en 3, 5 y 9 meses. Si la tasa de interés es del \textbf3.2\% mensual, ¿cuál será el valor de cada pago?

  \item La empresa Distribuidora Omega adquirió productos por \textbf{\$ 16,535.00} y planea pagarlos en tres cuotas iguales a 1, 6 y 11 meses. Si la tasa de interés es del \textbf3.2\% mensual, ¿cuánto deberá pagar en cada cuota?

  \item Una persona firma un pagaré por \textbf{\$ 7,309.00} a pagar en 4 meses, con una tasa del \textbf33.4\% anual. A los 2 meses contrae otra deuda de \textbf{\$ 9,640.00} a 3 meses. Luego, abona \textbf{\$ 4,617.00} y pacta el pago final a 6 meses del abono. ¿Cuánto debe pagar al final?

  \item Se depositan \textbf{₡145.661.00} en una cuenta con interés compuesto anual del \textbf17.5\%, convertible trimestralmente. ¿Cuál será el interés ganado al cabo de 1 año?

  \item Jorge invierte \textbf{\$ 34,574.00} al \textbf4.7\% anual durante 4 años. ¿Cuánto interés gana en ese periodo? Utilice interés compuesto.

  \item Se depositan \textbf{\$ 64,197.00} en una cuenta que paga \textbf2.0\% mensual, capitalizable mensualmente. ¿Cuál será el monto acumulado en 18 meses?
\end{enumerate}
\newpage

\begin{center}
    {\LARGE \textbf{II Examen de Matemática Financiera}}\\[1em]
    {\large \textbf{Estudiante:} ARIAS MARIN KENNETH JESUS}
\end{center}

\vspace{1cm}

\textbf{Profesor: Francisco Campos Sandi}

\textbf{Instrucciones:} Desarrolle con orden y claridad los siguientes ejercicios. Puede utilizar calculadora financiera. Fecha de entrega: \textbf{Miércoles 17 de Julio}.

\begin{enumerate}
  \item El señor Eduardo Morales realizó una compra de mercancía por \textbf{\$ 24,396.00}. Acordó hacer tres pagos iguales: en 3, 5 y 9 meses. Si la tasa de interés es del \textbf3.0\% mensual, ¿cuál será el valor de cada pago?

  \item La empresa Distribuidora Omega adquirió productos por \textbf{\$ 16,345.00} y planea pagarlos en tres cuotas iguales a 1, 6 y 11 meses. Si la tasa de interés es del \textbf4.0\% mensual, ¿cuánto deberá pagar en cada cuota?

  \item Una persona firma un pagaré por \textbf{\$ 7,595.00} a pagar en 4 meses, con una tasa del \textbf33.3\% anual. A los 2 meses contrae otra deuda de \textbf{\$ 9,931.00} a 3 meses. Luego, abona \textbf{\$ 3,381.00} y pacta el pago final a 6 meses del abono. ¿Cuánto debe pagar al final?

  \item Se depositan \textbf{₡149.990.00} en una cuenta con interés compuesto anual del \textbf16.3\%, convertible trimestralmente. ¿Cuál será el interés ganado al cabo de 1 año?

  \item Jorge invierte \textbf{\$ 33,615.00} al \textbf5.0\% anual durante 4 años. ¿Cuánto interés gana en ese periodo? Utilice interés compuesto.

  \item Se depositan \textbf{\$ 69,072.00} en una cuenta que paga \textbf2.0\% mensual, capitalizable mensualmente. ¿Cuál será el monto acumulado en 18 meses?
\end{enumerate}
\newpage

\begin{center}
    {\LARGE \textbf{II Examen de Matemática Financiera}}\\[1em]
    {\large \textbf{Estudiante:} FERNANDEZ FLORES YOSEF SAID}
\end{center}

\vspace{1cm}

\textbf{Profesor: Francisco Campos Sandi}

\textbf{Instrucciones:} Desarrolle con orden y claridad los siguientes ejercicios. Puede utilizar calculadora financiera. Fecha de entrega: \textbf{Miércoles 17 de Julio}.

\begin{enumerate}
  \item El señor Eduardo Morales realizó una compra de mercancía por \textbf{\$ 24,967.00}. Acordó hacer tres pagos iguales: en 3, 5 y 9 meses. Si la tasa de interés es del \textbf2.9\% mensual, ¿cuál será el valor de cada pago?

  \item La empresa Distribuidora Omega adquirió productos por \textbf{\$ 14,710.00} y planea pagarlos en tres cuotas iguales a 1, 6 y 11 meses. Si la tasa de interés es del \textbf3.4\% mensual, ¿cuánto deberá pagar en cada cuota?

  \item Una persona firma un pagaré por \textbf{\$ 6,420.00} a pagar en 4 meses, con una tasa del \textbf33.7\% anual. A los 2 meses contrae otra deuda de \textbf{\$ 9,565.00} a 3 meses. Luego, abona \textbf{\$ 3,758.00} y pacta el pago final a 6 meses del abono. ¿Cuánto debe pagar al final?

  \item Se depositan \textbf{₡143.254.00} en una cuenta con interés compuesto anual del \textbf14.4\%, convertible trimestralmente. ¿Cuál será el interés ganado al cabo de 1 año?

  \item Jorge invierte \textbf{\$ 34,994.00} al \textbf4.3\% anual durante 4 años. ¿Cuánto interés gana en ese periodo? Utilice interés compuesto.

  \item Se depositan \textbf{\$ 68,495.00} en una cuenta que paga \textbf2.3\% mensual, capitalizable mensualmente. ¿Cuál será el monto acumulado en 18 meses?
\end{enumerate}
\newpage

\begin{center}
    {\LARGE \textbf{II Examen de Matemática Financiera}}\\[1em]
    {\large \textbf{Estudiante:} LARA HERNANDEZ DYLAN DANIEL}
\end{center}

\vspace{1cm}

\textbf{Profesor: Francisco Campos Sandi}

\textbf{Instrucciones:} Desarrolle con orden y claridad los siguientes ejercicios. Puede utilizar calculadora financiera. Fecha de entrega: \textbf{Miércoles 17 de Julio}.

\begin{enumerate}
  \item El señor Eduardo Morales realizó una compra de mercancía por \textbf{\$ 23,452.00}. Acordó hacer tres pagos iguales: en 3, 5 y 9 meses. Si la tasa de interés es del \textbf3.1\% mensual, ¿cuál será el valor de cada pago?

  \item La empresa Distribuidora Omega adquirió productos por \textbf{\$ 16,566.00} y planea pagarlos en tres cuotas iguales a 1, 6 y 11 meses. Si la tasa de interés es del \textbf3.0\% mensual, ¿cuánto deberá pagar en cada cuota?

  \item Una persona firma un pagaré por \textbf{\$ 6,554.00} a pagar en 4 meses, con una tasa del \textbf25.2\% anual. A los 2 meses contrae otra deuda de \textbf{\$ 9,246.00} a 3 meses. Luego, abona \textbf{\$ 3,628.00} y pacta el pago final a 6 meses del abono. ¿Cuánto debe pagar al final?

  \item Se depositan \textbf{₡152.533.00} en una cuenta con interés compuesto anual del \textbf14.9\%, convertible trimestralmente. ¿Cuál será el interés ganado al cabo de 1 año?

  \item Jorge invierte \textbf{\$ 33,919.00} al \textbf4.2\% anual durante 4 años. ¿Cuánto interés gana en ese periodo? Utilice interés compuesto.

  \item Se depositan \textbf{\$ 63,423.00} en una cuenta que paga \textbf2.3\% mensual, capitalizable mensualmente. ¿Cuál será el monto acumulado en 18 meses?
\end{enumerate}
\newpage

\begin{center}
    {\LARGE \textbf{II Examen de Matemática Financiera}}\\[1em]
    {\large \textbf{Estudiante:} RUIZ ARAYA CRISTOPHER ANDREY}
\end{center}

\vspace{1cm}

\textbf{Profesor: Francisco Campos Sandi}

\textbf{Instrucciones:} Desarrolle con orden y claridad los siguientes ejercicios. Puede utilizar calculadora financiera. Fecha de entrega: \textbf{Miércoles 17 de Julio}.

\begin{enumerate}
  \item El señor Eduardo Morales realizó una compra de mercancía por \textbf{\$ 22,350.00}. Acordó hacer tres pagos iguales: en 3, 5 y 9 meses. Si la tasa de interés es del \textbf3.1\% mensual, ¿cuál será el valor de cada pago?

  \item La empresa Distribuidora Omega adquirió productos por \textbf{\$ 16,436.00} y planea pagarlos en tres cuotas iguales a 1, 6 y 11 meses. Si la tasa de interés es del \textbf3.7\% mensual, ¿cuánto deberá pagar en cada cuota?

  \item Una persona firma un pagaré por \textbf{\$ 6,238.00} a pagar en 4 meses, con una tasa del \textbf25.3\% anual. A los 2 meses contrae otra deuda de \textbf{\$ 10,487.00} a 3 meses. Luego, abona \textbf{\$ 4,284.00} y pacta el pago final a 6 meses del abono. ¿Cuánto debe pagar al final?

  \item Se depositan \textbf{₡149.624.00} en una cuenta con interés compuesto anual del \textbf14.6\%, convertible trimestralmente. ¿Cuál será el interés ganado al cabo de 1 año?

  \item Jorge invierte \textbf{\$ 33,381.00} al \textbf4.3\% anual durante 4 años. ¿Cuánto interés gana en ese periodo? Utilice interés compuesto.

  \item Se depositan \textbf{\$ 60,183.00} en una cuenta que paga \textbf2.2\% mensual, capitalizable mensualmente. ¿Cuál será el monto acumulado en 18 meses?
\end{enumerate}
\newpage

\begin{center}
    {\LARGE \textbf{II Examen de Matemática Financiera}}\\[1em]
    {\large \textbf{Estudiante:} VARGAS GUTIERREZ JAIRO}
\end{center}

\vspace{1cm}

\textbf{Profesor: Francisco Campos Sandi}

\textbf{Instrucciones:} Desarrolle con orden y claridad los siguientes ejercicios. Puede utilizar calculadora financiera. Fecha de entrega: \textbf{Miércoles 17 de Julio}.

\begin{enumerate}
  \item El señor Eduardo Morales realizó una compra de mercancía por \textbf{\$ 24,552.00}. Acordó hacer tres pagos iguales: en 3, 5 y 9 meses. Si la tasa de interés es del \textbf2.5\% mensual, ¿cuál será el valor de cada pago?

  \item La empresa Distribuidora Omega adquirió productos por \textbf{\$ 15,877.00} y planea pagarlos en tres cuotas iguales a 1, 6 y 11 meses. Si la tasa de interés es del \textbf3.4\% mensual, ¿cuánto deberá pagar en cada cuota?

  \item Una persona firma un pagaré por \textbf{\$ 6,450.00} a pagar en 4 meses, con una tasa del \textbf31.6\% anual. A los 2 meses contrae otra deuda de \textbf{\$ 9,432.00} a 3 meses. Luego, abona \textbf{\$ 3,718.00} y pacta el pago final a 6 meses del abono. ¿Cuánto debe pagar al final?

  \item Se depositan \textbf{₡157.475.00} en una cuenta con interés compuesto anual del \textbf17.4\%, convertible trimestralmente. ¿Cuál será el interés ganado al cabo de 1 año?

  \item Jorge invierte \textbf{\$ 34,175.00} al \textbf4.1\% anual durante 4 años. ¿Cuánto interés gana en ese periodo? Utilice interés compuesto.

  \item Se depositan \textbf{\$ 63,294.00} en una cuenta que paga \textbf2.4\% mensual, capitalizable mensualmente. ¿Cuál será el monto acumulado en 18 meses?
\end{enumerate}
\newpage

\begin{center}
    {\LARGE \textbf{II Examen de Matemática Financiera}}\\[1em]
    {\large \textbf{Estudiante:} VARGAS GUTIERREZ JOSUE ANDRES}
\end{center}

\vspace{1cm}

\textbf{Profesor: Francisco Campos Sandi}

\textbf{Instrucciones:} Desarrolle con orden y claridad los siguientes ejercicios. Puede utilizar calculadora financiera. Fecha de entrega: \textbf{Miércoles 17 de Julio}.

\begin{enumerate}
  \item El señor Eduardo Morales realizó una compra de mercancía por \textbf{\$ 23,158.00}. Acordó hacer tres pagos iguales: en 3, 5 y 9 meses. Si la tasa de interés es del \textbf3.5\% mensual, ¿cuál será el valor de cada pago?

  \item La empresa Distribuidora Omega adquirió productos por \textbf{\$ 14,527.00} y planea pagarlos en tres cuotas iguales a 1, 6 y 11 meses. Si la tasa de interés es del \textbf3.1\% mensual, ¿cuánto deberá pagar en cada cuota?

  \item Una persona firma un pagaré por \textbf{\$ 7,422.00} a pagar en 4 meses, con una tasa del \textbf32.7\% anual. A los 2 meses contrae otra deuda de \textbf{\$ 10,283.00} a 3 meses. Luego, abona \textbf{\$ 3,961.00} y pacta el pago final a 6 meses del abono. ¿Cuánto debe pagar al final?

  \item Se depositan \textbf{₡150.236.00} en una cuenta con interés compuesto anual del \textbf15.0\%, convertible trimestralmente. ¿Cuál será el interés ganado al cabo de 1 año?

  \item Jorge invierte \textbf{\$ 31,382.00} al \textbf4.8\% anual durante 4 años. ¿Cuánto interés gana en ese periodo? Utilice interés compuesto.

  \item Se depositan \textbf{\$ 67,493.00} en una cuenta que paga \textbf2.3\% mensual, capitalizable mensualmente. ¿Cuál será el monto acumulado en 18 meses?
\end{enumerate}
\newpage
\end{document}